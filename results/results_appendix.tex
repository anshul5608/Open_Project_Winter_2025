\section{Results Appendix}

\subsection{Experimental Configuration}

The model was trained to reconstruct single-qubit density matrices 
from Pauli expectation values:

\[
\langle X \rangle, \quad \langle Y \rangle, \quad \langle Z \rangle
\]

Each training sample corresponds to a randomly generated physically valid 
single-qubit quantum state.

The neural network is a fully connected multilayer perceptron (MLP) 
that outputs parameters of a lower triangular matrix $L$. 
The density matrix is constructed as:

\[
\rho = \frac{L L^\dagger}{\mathrm{Tr}(L L^\dagger)}
\]

This parameterization guarantees:

\begin{itemize}
    \item Hermiticity
    \item Positive Semi-Definiteness
    \item Unit Trace
\end{itemize}

Physical validity is therefore enforced by construction rather than post-processing.

Training and evaluation were performed using a classical CPU-based pipeline.

---

\subsection{Evaluation Metrics}

\subsubsection{Quantum Fidelity}

Quantum fidelity between predicted density matrix $\rho$ and 
true density matrix $\sigma$ is defined as:

\[
F(\rho, \sigma) =
\left(
\mathrm{Tr}
\sqrt{
\sqrt{\rho} \, \sigma \, \sqrt{\rho}
}
\right)^2
\]

Fidelity ranges from 0 to 1, with $F=1$ indicating identical quantum states.

---

\subsubsection{Trace Distance}

Trace distance between $\rho$ and $\sigma$ is defined as:

\[
D(\rho, \sigma) =
\frac{1}{2}
\left\|
\rho - \sigma
\right\|_1
\]

where $\|\cdot\|_1$ denotes the trace norm.

Trace distance measures distinguishability between two quantum states.
Lower values indicate closer agreement.

---

\subsection{Numerical Performance Summary}

\input{performance_table.tex}

The model achieves high mean fidelity and low trace distance, 
demonstrating accurate reconstruction of single-qubit quantum states.

Inference latency remains sub-millisecond, 
making the approach suitable for efficient classical post-processing 
in hybrid quantum-classical workflows.

---

\subsection{Distribution Analysis}

\begin{figure}[h]
\centering
\includegraphics[width=0.65\linewidth]{fidelity_hist.png}
\caption{Distribution of quantum fidelity values across the evaluation dataset. 
The concentration of values near unity indicates high reconstruction accuracy.}
\end{figure}

\begin{figure}[h]
\centering
\includegraphics[width=0.65\linewidth]{trace_distance_hist.png}
\caption{Distribution of trace distance values. 
Most values cluster near zero, confirming strong agreement between predicted and true density matrices.}
\end{figure}

---

\subsection{Behavioural Observations}

Several important observations emerge from the experiments:

\begin{itemize}
    \item The Cholesky-based parameterization stabilizes training and prevents non-physical outputs.
    \item Fidelity values are tightly concentrated near 1, indicating robust reconstruction.
    \item Trace distance remains consistently low across the evaluation set.
    \item No post-hoc projection or eigenvalue correction was required.
\end{itemize}

The enforcement of quantum mechanical constraints directly in the model 
significantly improves training stability compared to unconstrained regression approaches.

---

\subsection{Scaling Considerations}

For an $n$-qubit system, the Hilbert space dimension scales as:

\[
\dim(\mathcal{H}) = 2^n
\]

The density matrix contains:

\[
4^n
\]

real parameters.

This exponential growth leads to:

\begin{itemize}
    \item Rapid increase in model output dimensionality
    \item Increased dataset size requirements
    \item Higher computational and memory cost
\end{itemize}

While the present model performs efficiently for single-qubit systems,
extension to multi-qubit systems requires architectural refinement,
efficient measurement schemes, and improved scalability strategies.

---

\subsection{Conclusion of Results}

The experimental results demonstrate that a classical neural network,
combined with physics-informed parameterization, can accurately reconstruct
physically valid single-qubit quantum states.

The approach achieves:

\begin{itemize}
    \item High reconstruction fidelity
    \item Low trace distance
    \item Stable optimization behaviour
    \item Efficient inference performance
\end{itemize}

These results validate the feasibility of machine learning–based
quantum state tomography for low-dimensional quantum systems.
